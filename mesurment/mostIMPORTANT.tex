\subsection{Smallest detectable object}\label{subsection:mesurment-smalest}
The smallest object that the sensor can detect is dependent on zise of the object, speed of the LIDAR and distance between the LIDAR and the object.

In our test we tested with an round object that had a diameter of $6mm$. 
After a series of test the conclusion was that it could see the object in between $113mm$ and $300mm$ when the delay was $40ms$ in between each step.
If the step delay adjusted down to $1ms$ the LIDAR was still able to detect it sometimes and the range decreased to below $230mm$.


\subsection{Minimum speed}\label{subsetion:mesurmen-miniSpeed}
The minimum speed was also dependent on in with direction the object travelled in relation to the LIDAR. 
Al test was done with an one meter long wooden plank that was hanging from a string so it could be easy manipulated.
When the object approached the LIDAR that was rotating with an delay of $1.5ms$ against the LIDARs beam the measurement showed that the object wasent detecteble if the object moved faster then $1m/s$. 
The other direction gave the result $0.4m/s$ dependent a bit of the initial position.

\subsection{Different objects}\label{subsection:mesurment-difObj}
The test that was preformed was if the LIDAR could see an transparent acrylic box.
And the data suggested it wasn't any big difference between the box and the solid object. 
Highly reflective material was only captured when it was almost orgogonal to the object.